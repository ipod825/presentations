\documentclass[compress]{beamer}

% \mode<presentation>{
\usetheme{Singapore}
\usecolortheme{rose}
% }
% \pgfplotsset{/pgf/number format/use comma,compat=newest}
% \usepackage{color}
\usepackage{amsmath,amsfonts,amssymb}
\usepackage{CJKutf8}
% \usepackage{hyperref}
% \usepackage{tikz}

\title{ANTUSD: A Large Chinese Sentiment Dictionary }
\author{Shih-Ming Wang and Lun-Wei Ku}
\institute{NLPLab, Institute of Information Science, Academia Sinica}
% \date{\today}
\date{May 25, 2016}
\subject{Computer Science}
\begin{document}
\beamertemplatenavigationsymbolsempty

\begin{frame}
    \maketitle
\end{frame}

\begin{frame}
    \frametitle{Outline}
    \tableofcontents
\end{frame}
\section{Motivation}
    \begin{frame}{\secname}
        \begin{itemize}
            \item Sentiment dictionary
            \begin{itemize}
                \item A building block of sentiment analysis \& opinion mining
                % \item A useful resource for both research and industrial communities
                \item Applied as markers or machine learning features
            \end{itemize}
            \item Augmented NTU\footnote{The original authors of NTUSD were researchers at National Taiwan University} Sentiment Dictionary (ANTUSD)
            \begin{itemize}
                \item Lack of Chinese resource
                \item Big \& complete
                \item Expert labeled sentiment \& machine predicted sentiment scores
            \end{itemize}
        \end{itemize}
    \end{frame}


\begin{CJK*}{UTF8}{bsmi}
\section{Related Work}
    \subsection{Corpora Resource}
        \begin{frame}{\subsecname\ I}
            \begin{itemize}
                \item Words and labels were collected from several sentiment corpora (2006$\sim$2010)
                \item Word-base, context free
                \begin{itemize}
                    \item NTUSD
                        \begin{itemize}
                            \item A widely used Chinese sentiment dictionary
                            \item Labels: \textbf{POS} and \textbf{NEG} (2812/8276) 
                        \end{itemize}
                    \item ACIBiMA\footnote{Advanced Chinese Bi-Character Word Morphological Analyzer}
                        \begin{itemize}
                            \item Built to test Chinese morphological structure and sentiment
                            \item Labels: \textbf{POS}, \textbf{NEU}, \textbf{NEG}, \textbf{NONOP}, and \textbf{NOT}
                            \item \textbf{NONOP} consists of regular non-emotion words
                            \item \textbf{NOT} consists of incorrectly segmented words
                        \end{itemize}
                \end{itemize}
            \end{itemize}
        \end{frame}

        \begin{frame}{\subsecname\ II}
            \begin{itemize}
                \item Sentence-based, context dependent
                    \begin{itemize}
                        \item NTCIR\footnote{http://research.nii.ac.jp/ntcir/index-en.html} MOAT Task Dataset 
                            % A internation competition related to opinion analysis 
                            \begin{itemize}
                                \item Annotators labeled each sentences as \textbf{Positive}, \textbf{Neutral}, and \textbf{Neutral}
                                \item Annotators also labeled sentiment words as they read sentences
                                \item Label count for each word is it realted to its frequency
                            \end{itemize}
                        \item Chinese Opinion Treebank
                            % Invented to incorporate the syntactic information, such as parsing trees into sentiment analysis
                            \begin{itemize}
                                \item The labeling process is similar to that of NTCIR
                            \end{itemize}
                    \end{itemize}
            \end{itemize}
        \end{frame}
        
        \begin{frame}{\subsecname\ III}
            \begin{itemize}
                \item ANTUSD \& ACIBiMA has single gold label for each word
                \item NTCIR \& Chinese Opinion Treebank has mutiple (possibly conflicting) labels for each word
            \end{itemize}
            
        \end{frame}

\section{Dictionary Schema}
    \begin{frame}{\secname}
        \begin{itemize}
            \item ANTUSD \& ACIBiMA has single 
        \end{itemize}
        
    \end{frame}
\end{CJK*}

\section{Demonstrative Experiment}


\end{document}
