\documentclass[compress]{beamer}

\mode<presentation>
\usetheme{Singapore}
\usecolortheme{rose}

% \usepackage{amsmath,amsfonts,amssymb}
% \usepackage{colortbl}
% \usepackage{hyperref}
% \usepackage{tikz}
% \usetikzlibrary{arrows,decorations.pathmorphing,backgrounds,positioning,fit,petri}


%%% Template Setting

%% Font
\usefonttheme[onlylarge]{structuresmallcapsserif}
\usefonttheme[onlysmall]{structurebold}
\setbeamerfont{title}{shape=\itshape,family=\rmfamily}
\setbeamertemplate{frametitle}[default][center]


\title{Molding CNNs for text: non-linear, non-consecutive convolutions}
\author{Tao Lei, Regina Barzilay, and Tommi Jaakkola}
\institute{Presented by Shih-Ming Wang \\ NLPLab, Institute of Information Science, Academia Sinica}
\date{07-5-2016}
\subject{Computer Science}

\graphicspath{{img/}}

\begin{document}
\beamertemplatenavigationsymbolsempty

\begin{frame}
 \maketitle
\end{frame}

\begin{frame}
 \frametitle{Outline}
 \tableofcontents
\end{frame}
\section{Introduction}

\section{Motivation}
\begin{frame}{\secname}
    \begin{itemize}
        \item Deep learning \& Convolution neural network (CNN) have led to success in many NLP problems
        \item Convolution operation is a \textbf{linear} mapping over \textbf{n-gram} vectors
        \item Target: \textbf{non-linear} operation over \textbf{non-consecutive} n-grams (e.g., ``\underline{not} that \underline{good}'')
    \end{itemize}
\end{frame}

\section{Background}
\begin{frame}{\secname}
    
\end{frame}

\section{Model Description}
\subsection{Tensor-based Feature Mapping}
\begin{frame}{\subsecname}
    \begin{itemize}
    \item Intuition: use product operation to remedy the insufficiency of linear operation
    \item Consider 2-gram $(x_1, x_2)$ as example:
    \end{itemize}
    \begin{table}[t]
        \centering
        \label{compare}
        \begin{tabular}{ccc}
                                       & Linear                  & Product     \\ \hline
        Dim($x$)                       & $1\times d$             & $1\times d$ \\ \hline
        Dim(filter) $h\times2\times d$ & $h\times 2 \times d^2$ \\ \hline
        Output                         & $1\times h$             & $h \times d^2$ \hline

        \end{tabular}
    \end{table}
\end{frame}

\subsection{Non-consecutive n-gram Features}
\begin{frame}{\secname}
    
\end{frame}


\section{Experiments}
\begin{frame}{\secname}
\end{frame}

\section{Error Analysis}
\begin{frame}{\secname}
\end{frame}



\end{document}
